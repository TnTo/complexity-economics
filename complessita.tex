% !TeX spellcheck = it_IT
\documentclass[a4paper, headings=standardclasses]{scrartcl}

\usepackage[margin=2.5cm]{geometry}
\usepackage{authblk}
\renewcommand{\Affilfont}{\small}
\usepackage[style=apa, backend=biber, sorting=ynt, sortcites=true, useprefix=true]{biblatex}
\usepackage[autostyle=false, style=english]{csquotes}
\MakeOuterQuote{"}
\usepackage[italian]{babel}
\usepackage[modulo]{lineno}
\linenumbers
\usepackage[hidelinks]{hyperref}

\usepackage{textcomp}
\usepackage{extdash}

\geometry{
	marginparsep=1em,
	marginparwidth=1.75cm
}
\NewCommandCopy{\oldmarginpar}{\marginpar}
\RenewDocumentCommand{\marginpar}{om}{%
	\IfNoValueTF{#1}
	{\leavevmode\oldmarginpar{\mymparsetup #2}}
	{\leavevmode\oldmarginpar[\mymparsetup #1]{\mymparsetup #2}}}
\newcommand{\mymparsetup}{\itshape \raggedright \scriptsize}


\addbibresource{complexity.bib}

%opening
\title{Sulla complessità come meta-teoria\let\thefootnote\relax\footnotetext{
		Versioni precedenti di questo lavoro sono state presentate alla 24esima ESHET Summer School e alla 2023 INEM Conference. \\
		L'ultima versione di questo lavoro è disponibile online \url{https://github.com/TnTo/complexity-economics/}.
}}
\subtitle{una discussione dalla prospettiva dell'economia}
\author{Margherita Redigonda\thanks{m@orsorosso.net - \url{https://orcid.org/0000-0003-1485-1204}}}

\begin{document}

\maketitle

\begin{abstract}
	L'assenza di una definizione condivisa di complessità rende difficile ogni discorso su di essa.
	Questo lavoro prova a fornire una definizione di complessità seguendo l'intuizione espressa da Anderson nell'articolo del 1972 "More is Different" in senso anti-riduzionista e riconoscendo la differenza tra l'idea di complessità e i metodi matematici storicamente usati per lo studi dei sistemi complessi.
	La definizione fornita permette di riconoscere la complessità come meta-teoria piuttosto che come teoria scientifica, descrivendone le caratteristiche.
	Successivamente la relazione tra complessità ed economia viene esplorata da una prospettiva storica e metodologica, riconoscendo tre diversi archetipi di relazione (neo-empirica, post-neoclassica, meta-teorica).
	Infine, vengono descritte quali norme sociali ostacolano o faciliterebbero l'adozione della meta-teoria della complessità nella pratica scientifica. \\
	\textbf{Keywords:} Complessità, Meta-Teoria, Riduzionismo\\
	\textbf{JEL Codes: B41, B59, A14}
\end{abstract}

\section{Sulla Complessità}
\marginpar{La natura sfuggente della complessità}Trattare il tema della complessità è generalmente qualcosa di molto difficile da fare per la natura ambigua e sfuggente del termine.
Innanzitutto, complessità è ormai diventata una \textit{buzzword} spesso priva di significato proprio.
Inoltre non esiste una definizione condivisa di cosa sia la complessità a livello scientifico\footnote{\textcite{horgan2015}, citato in \textcite{holt2011}, elenca 45 differenti idee da cui partire per cercare una definizione di complessità.} rendendo difficile trattare con precisione l'argomento.

La causa di ciò credo divenga evidente una volta che si guardi il problema della definizione di complessità da una prospettiva storica.

Una lista di ciò che oggi è senza dubbio oggetto di studio da parte della teoria della complessità non può non contenere, tra le altre cose, i sistemi dinamici (in particolare il caos deterministico e la teoria delle biforcazioni) e la meccanica statistica. Per entrambe le discipline di studio è possibile ricondurne la nascita (o almeno un momento fondativo) agli anni 1870, quando Poincaré pubblicò i primi lavori sulle ricorrenze e Boltzmann presentò i primi risultati su cui fondare la meccanica statistica.

D'altra parte, il termine complessità entra nel lessico scientifico con l'articolo di \citeauthor{anderson1972} del \citeyear{anderson1972} \citetitle{anderson1972}, che si apre mettendo in discussione il paradigma riduzionista su cui si basa la scienza moderna, riconoscendo che sebbene si possa creare una gerarchia di scienze secondo la visione riduzionista (ad esempio fisica \textrightarrow{} chimica \textrightarrow{} biologia \textrightarrow{} fisiologia \textrightarrow{} psicologia \textrightarrow{} sociologia) non è sostenibile affermare che "X è semplicemente Y applicata".
Ovvero che non è possibile riformulare sulla base delle leggi della disciplina più ridotta tutte le leggi della disciplina più complessa (cioè non possiamo esprimere, ad esempio, tutta la conoscenza psicologica facendo riferimento solo ai fenomeni fisiologici sottostanti).

Nel momento in cui Anderson pubblica il suo articolo, però, sono già stati sviluppati numerosi metodi matematici per descrivere i "sistemi complessi" e si è già diffuso un \textit{senso comune} su quali problemi siano oggetto di studio da parte della teoria della complessità (come citato prima i sistemi dinamici e la termodinamica, a cui possiamo aggiungere, senza pretesa di esaustività, la teoria dell'informazione, i processi stocastici, i modelli computazionali ad agenti, la teoria dei giochi, la teoria dei grafi o delle reti).
In altre parole, nel momento in cui compare una definizione e un tentativo di fondare metodologicamente il campo di ricerca, lo sviluppo delle tecniche matematiche per indagarlo è già avanzato e si è sviluppato in una prospettiva indipendente e riduzionista.

Compare qui una tensione che è probabilmente il punto focale dell'articolo: seguendo Anderson la complessità non può che essere anti-riduzionista ma i metodi matematici della teoria della complessità nascono e si sviluppano in una prospettiva riduzionista.

Torniamo alla meccanica statistica come esempio archetipico. Il problema che essa risolve è di ricondurre un fenomeno macroscopico (le leggi della termodinamica e l'irreversibilità dei processi fisici) alle leggi microscopiche note (la dinamica molecolare) come prescritto dal paradigma riduzionista.  Ciò elimina la necessità di avere due differenti teorie per il microscopico e il macroscopico, riconducendo i fenomeni macroscopici a manifestazioni di leggi microscopiche.

Questo in teoria. Nella pratica, però, le leggi macroscopiche rimangono correntemente in uso per ragioni di semplicità e adeguatezza, perché descrivono meglio la realtà macroscopica e restituiscono rappresentazioni più semplici da usare e che sono comunque sufficientemente accurate.
Inoltre, la meccanica statistica ha creato una nuova descrizione mesoscopica dove le leggi microscopiche e macroscopiche si mischiano descrivendo fenomeni che non possono essere spiegati né solo dalle une né solo dalle altre, richiedendo di fatto un apparato teorico nuovo di leggi mesoscopiche.

In questo senso possiamo recuperare l'intuizione di Anderson: nel momento in cui si prova a ridurre il macroscopico al microscopico si perde in accuratezza nella descrizione del macroscopico (o almeno di praticità) mentre si esplora un'area grigia che non appartiene né al macroscopico né al microscopico e, al contempo, appartiene a entrambi.
Si ottiene quindi una nuova formulazione che è nella pratica inutilizzabile e inadeguata a essere usata come teoria scientifica per uno dei due livelli e, potenzialmente, si mette in luce un livello intermedio terzo che ha bisogno di un proprio e distinto apparato di leggi.

Per procedere oltre nella discussione è probabilmente necessario quantomeno avvicinarsi a una definizione dell'oggetto di studio: la complessità.

I concetti che più spesso vengono associati a quest'idea sono la relazione tra le parti e il tutto, la differenza tra comportamenti individuali e collettivi, le proprietà emergenti, le relazioni tra le parti, la non-linearità\footnote{La non-linearità potrebbe sembrare la carta spaiata nell'elenco, ma la sua relazione con le altre idee è semplice da mostrare. Ipotizziamo un insieme di elementi $\{x_i\}$ e supponiamo di aggregarli sommandoli come $X=\sum_i x_i$. La relazione tra il cambiamento di uno degli elementi dell'insieme e il cambiamento dell'aggregato è $\Delta X=\Delta x_i$, identificando tra loro i cambiamenti macro e microscopici ed eliminando la necessità di due rappresentazioni diverse della dinamica. Un caso forse più comune e analogo in economia è quello della log-linearità quando la funzione di aggregazione è il prodotto, ovvero $\log X = \log(\prod_i x_i)$ e $\Delta \log X=\Delta\log(x_i)$.}. Per questo motivo è comune parlare di \textit{sistemi complessi} quasi come sinonimo di complessità, perché la parola sistema sottende non molto più che un insieme di parti in relazione tra loro.

% Il confronto con altre definizioni può essere espanso
\marginpar{Una definizione}La definizione di complessità\footnote{Il problema della definizione di complessità, con particolare riguardo alla disciplina economica, è affrontato da \textcite[§3][]{holt2011} che propongono tre alternative rifiutando però la necessità di una sintesi.} che propongo è la seguente:
\begin{quote}
	Un sistema è complesso se deve essere descritto differentemente su diversi livelli di una o più scale.
\end{quote}

Per assumere di senso, questa definizione, richiede di definire cosa siano una scala e i suoi livelli. Ma prima di fare ciò spenderò due parole sull'idea di descrizione.

Ogni teoria scientifica mira alla comprensione del reale attraverso una sua rappresentazione semplificata che permetta di metterne a fuoco particolari caratteristiche d'interesse. Queste rappresentazioni sono generalmente prodotte nella forma di leggi o modelli.
Al contempo però, ciascuna di queste rappresentazioni esprime solo una particolare descrizione del reale che si concentra su alcuni dettagli tralasciandone altri\footnote{Si potrebbe, come esperimento mentale, pensare di riuscire a creare una descrizione del reale così comprensiva e precisa, al contempo analitica e sintetica, da rendere ogni altra obsoleta, ma non si discosterebbe molto da una mappa 1:1 del mondo, sulla cui inutilità e impraticità hanno scritto, meglio di me, Eco e Borges.}.

Questa osservazione ci consegna una prima intuizione su cosa sia la complessità: riconoscere che il mondo è composto da troppi enti, legati tra loro da troppi nessi e relazioni, per poterne isolare uno alla volta o studiarli tutti insieme con precisione. Invece, è necessario riconoscere, di volta in volta, quali siano i dettagli importanti messi a fuoco, l'ingrandimento necessario per vedere ciò che interessa, consapevoli però di cosa e perché sia rimasto fuori dal campo visivo.

Seguendo questa metafora provo a spiegare cosa intendo per scala e livelli.

Pensiamo a un vetrino per il microscopio con un singolo campione. Questo campione appare
in maniera molto diversa a seconda dell'ingrandimento o del piano focale scelti per l'osservazione.
Queste due variabili, due dimensioni lungo cui muoversi, cambiano il nostro modo di osservare il campione e la descrizione che ne possiamo dare.
La percezione che abbiamo del campione e quindi le caratteristiche e le proprietà che possiamo
descrivere, variano al variare delle due variabili (l'ingrandimento e il piano focale) di osservazione.

Abbiamo cioè introdotto due dimensioni, che chiamiamo scale, che presentano differenti modi, i livelli, di osservare, o descrivere, uno stesso ente.

Esempi tipici di scale sono la scala geografica (che in riferimento al sistema economico ha come livelli, ad esempio, una città, un distretto industriale, una nazione, un continente, l'intero
mercato globale) oppure la scala temporale (tra i cui livelli possiamo elencare il breve periodo, usato ad esempio per i modelli a capitale fisso, e il lungo periodo).

Un'altra scala, forse meno intuitiva, che trova enorme spazio nella descrizione dei fenomeni economici è la scala di aggregazione, tra i cui livelli troviamo l'individuo (micro), i gruppi (meso) e l'intera società (macro), che ci permette di descrivere come i concetti di molteplicità e relazione influenzano la descrizione del sistema economico, e quindi di descrivere quei comportamenti dell'individuo che trovano spiegazione solo nella sua relazione con altri.

Altre dimensioni lungo le quali varia la descrizione, e che quindi qui chiamo scale, sono meno intuitivamente delle scale.
Ad esempio possiamo interpretare la prospettiva di genere come una scala.
Uno stesso fenomeno può essere studiato ignorando il genere dei soggetti coinvolti, utilizzando una prospettiva binaria basandosi sul sesso biologico dei soggetti coinvolti, mantenendo una prospettiva comunque binaria ma basata sul genere e sulla socializzazione dei soggetti coinvolti, o adottando una prospettiva queer includendo una molteplicità di categorie di classificazione basate sull'autorappresentazione dei soggetti coinvolti.
Ognuno di questi livelli restituisce all'osservatore differenti caratteristiche del sistema studiato e nessuno di questi è a priori quello corretto.
Un progetto di ricerca in medicina sulla salute riproduttiva probabilmente utilizzerà una prospettiva basata sul sesso biologico che è, almeno in prima approssimazione, un fattore determinante, senza introdurre però una ricchezza maggiore di dettagli che risulterebbero solamente rumore all'atto dell'analisi statistica dei risultati\footnote{Almeno in prima approssimazione. Ottenuti i risultati dello studio iniziale potrebbe risultare estremamente utile cambiare livello di osservazione sulla scala per poter descrivere, ad esempio, gli stessi fenomeni nella popolazione sottoposta a terapia ormonale sostitutiva a seguito di una diagnosi di incongruenza di genere.}.
Allo stesso modo, un'etnografia sui movimenti queer difficilmente potrà rinunciare al livello di dettaglio ed eterogeneità che risiede nell'autorappresentazione dei singoli.

Riassumendo, una scala è un aspetto del sistema, una sua area semantica o concettuale, che possa essere analizzato da punti di vista diversi o con differenti livelli di dettaglio.
E un sistema complesso è qualunque sistema che cambi le proprie caratteristiche, o meglio la descrizione che se ne si può fare, a seconda del punto o del modo di osservazione (o del soggetto osservante). In altre parole, un sistema che non si mantenga sempre uguale a se stesso e coerente da ogni punto di vista.

È probabile che seguendo questa definizione la realtà sia complessa e sia complesso anche quasi ogni suo sottoinsieme. Ma ciò non significa che ogni modello, ogni rappresentazione di un aspetto del reale, debba fare ricorso ai metodi matematici della complessità o svilupparsi su più livelli di una scala.
Ci sono domande che possono ottenere risposta (o almeno una prima risposta sintetica e povera di dettagli) su un singolo livello su ogni scala, senza dover (o voler) considerare il mutamento del sistema al cambiare del sistema di osservazione.

Questa definizione, oltre a includere --credo-- tutte le differenti intuizioni sulla natura della complessità grazie all'astrattezza e alla generalezza dell'idea di scala, pone indirettamente l'attenzione sulle approssimazioni che esplicitamente o (più spesso) implicitamente vengono fatte in ogni ricerca scientifica e in ogni descrizione del reale.

\marginpar{Una meta-teoria}Quello che non fa questa definizione è fornire direttamente un qualche tipo di conoscenza specifica in qualsivoglia ambito del sapere.
In questo senso non possiamo considerare la teoria della complessità, per come è appena stata definita, una teoria in senso stretto.

Piuttosto essa fornisce delle indicazioni su come operazionalizzare l'osservazione, e quindi lo studio, di un sistema complesso, ovvero su come costruire delle teorie sui diversi sistemi complessi.

In questo senso credo sia più corretto parlare di una meta-teoria della complessità, ovvero di una teoria su come sia necessario sviluppare le teorie che descrivono i sistemi complessi\footnote{Se è vero quanto accennato prima sull'essere la realtà stessa un sistema complesso in ogni suo aspetto, allora la meta-teoria della complessità fornisce dei principi di cui deve tenere conto ogni teoria scientifica.}.

Si potrebbe argomentare che il caso particolare della fisica sia sufficiente ad abbattere la costruzione argomentativa che sto portando avanti, fornendo un fortissimo argomento a favore del mantenimento di un paradigma riduzionistico.
Attraverso passaggi logici è possibile a oggi ricondurre quasi ogni legge fisica a un piccolo numero di forze fondamentali (tra una e tre a seconda delle teorie), che sono in grado quindi di spiegare ogni fenomeno del reale.

Da un punto di vista speculativo, assumendo per semplicità una realtà assolutamente deterministica, questo potrebbe sembrare effettivamente la realizzazione del sogno riduzionista di ricondurre ogni fenomeno a una manciata di principi primi e quindi alla possibilità di formulare una \textit{legge del tutto} che governa la realtà in ogni suo aspetto. Ma tale pensiero non trova riscontro pratico o esperienziale.
Nessun ricercatore proverebbe mai a descrivere nemmeno la statica di un ponte usando le poche leggi fondamentali, per non dire di fenomeni più complessi nel regno animale o nella sfera culturale.
E come già detto, anche nella stessa fisica esistono fenomeni che sono ricondotti alla teoria unificante solo in condizioni ideali e perfette, le cui imprecisioni reali sono meglio spiegate da teorie e correzioni ad hoc, sviluppate per il caso specifico senza pretese di universalità.

L'anti-riduzionismo presente nel lavoro di Anderson e nell'idea di una meta-teoria della complessità può essere quindi riformulato come il rifiuto della possibilità di una teoria del tutto e di un punto di vista privilegiato.

Come conseguenza, \textbf{la conoscenza non può che essere intesa come contestuale e funzionale}, ovvero legata a particolari premesse e scopi che evidenziano caratteristiche diverse dello stesso oggetto di studio, determinando approssimazioni diverse che evidenziano diversi livelli su diverse scale.

Fare ricerca secondo questa accezione di complessità richiede quindi di rimettere al centro la specifica domanda di ricerca, o di riconoscere la specifica sfaccettatura del reale da osservare, e assumerla come punto di partenza.
Da lì è necessario descrivere nel modo più preciso e ricco possibile l'oggetto di studio per poter riconoscere quante più scale rilevanti possibili e su ognuna di esse i livelli più adatti allo scopo, ovvero di esplicitare con la maggiore precisione possibile le proprie premesse, per poi approssimare l'oggetto di studio a una sua rappresentazione --a un suo modello-- che sia gestibile e affrontabile.

\marginpar{Soggettivo, intersezionalità e descrizione}L'atto, qui così centrale, di descrivere richiede un'analisi accurata, senza, almeno in primo luogo, scorciatoie e approssimazioni. In alcuni contesti è possibile che ciò possa essere fatto con cura e minuzia usando il linguaggio formale della matematica o una lingua che non si padroneggia perfettamente, ma in generale se descrivere (e quindi, forse, comprendere intimamente) diventa un aspetto fondamentale della pratica scientifica, il ritorno all'uso del proprio linguaggio naturale (e forse anche l'uso della multi-medialità) diventa una pratica imprescindibile\footnote{Questa è solo una delle difficoltà pratiche che l'adozione della meta-teoria della complessità deve affrontare nella ricerca contemporanea. L'ultima sezione di questo articolo cerca di approfondire maggiormente la questione.}.

Così delineata, la meta-teoria della complessità si avvicina a una metodologia del particolare e dell'unico, che riconosce che i tratti importanti di un sistema sono molteplici e differenti a seconda dello scopo, della storia e della soggettività di chi porta avanti la ricerca. Riconosce, in una certa misura, il ruolo organico del particolare all'interno del generale, piuttosto che assumere il particolare come variazione sul tema del generale.

Il protagonismo del particolare richiede di estendere i metodi a disposizione del ricercatore, e in particolare di recuperare i metodi qualitativi anche dove il loro uso si è perso. Questo perché l'uso complementare di metodi qualitativi e quantitativi è in grado di esplorare lo stesso fenomeno a un maggior numero livelli diversi, per ottenerne una conoscenza più ricca e sfaccettata. Permette cioè di catturare anche quegli aspetti di un sistema che sono o difficilmente esprimibili attraverso il linguaggio astratto e formale della matematica\footnote{O, forse più propriamente, per cui la descrizioni ottenuta attraverso il linguaggio astratto e formale della matematica non aggiunge né chiarezza né precisione alla descrizione fatta con altri linguaggi, come quello naturale. Teoricamente, ogni descrizione fatta in linguaggio naturale può essere riscritta in maniera equivalente per contenuto con qualche forma di logica del primo ordine, ma solo in alcuni casi ciò risulta uno strumento chiarificatore delle forme del discorso anziché un modo di nascondere il contenuto dietro la forma.} o che riguardano esperienze sufficientemente rare o variegate da non poter ottenere per essi delle misure numerose e consistenti che permettano di soddisfare le assunzioni statistiche.

Allo stesso modo, l'esperienza umana individuale viene riconosciuta come portatrice una dimensione propria, quantomeno come determinante della descrizione che il ricercatore fa, osservabile dal giusto livello della giusta scala, e non come mera variazione di un archetipo generale.
In questo senso, penso si possa riconoscere che la complessità così intesa offra una cornice concettuale per quelle istanze che cercano di portare una prospettiva intersezionale, democratica e di cura all'interno della pratica scientifica.

\marginpar{Sul pluralismo}Per chiudere la sezione mi soffermo brevemente su di un dibattito vivo in molte discipline, tra cui l'economia, sulla coesistenza di teorie alternative e il problema di scegliere una teoria rispetto a un'altra. Il dibattito sul pluralismo.

Col termine pluralismo si indica di solito una visione per cui all'interno di una disciplina possano (o debbano) coesistere una pluralità di teorie diverse tra loro alternative.
L'approccio che stiamo delineando fornisce una possibile giustificazione al pluralismo.
Siccome ogni teoria nasce e si sviluppa in un preciso e limitato ambito del sapere, per spiegare fenomeni solo su alcuni livelli di alcune scale, è necessario ricorrere a teorie differenti qualore si esca dal dominio di validità.

Al contempo la fisica newtoniana, l'elettromagnetismo classico e persino il geocentrismo non sono in assoluto teorie sbagliate, ma semplicemente teorie e modelli che hanno un dominio di validità non universale, ma che nel loro dominio di validità rimangono utili perché rispondono alle domande per cui sono state create in modo semplice.

Un esempio dall'economia può essere invece lo studio dei fenomeni di lungo periodo.
Esistono almeno quattro approcci diversi nella storia dell'economia per descrivere il lungo periodo: l'approccio neoclassico, che descrive l'esistenza di un unico equilibrio di lungo periodo e il comportamento del sistema economico che rilassa verso di esso; l'approccio post-keynesiano, che descrive la possibilità di stati stazionari multipli, simili nell'idea a equilibri instabili, fra cui il sistema economico può muoversi; l'approccio marxiano, che descrive qualitativamente alcune caratteristiche della dinamica del sistema economico nel lungo periodo, senza descrivere però il punto di arrivo di essa con precisione sufficiente a essere studiato; l'approccio che fa propria la teoria del caos deterministico, che descrivendo l'economia come un sistema caotico conclude l'impossibilità di studiarne il comportamento di lungo periodo, per il progressivo accumularsi di errori inevitabili in qualunque rappresentazione.
Ciascuno di questi quattro approcci permette di studiare aspetti diversi del futuro, rispondendo a domande diverse.

La meta-teoria della complessità può dare supporto al pluralismo sostituendo alla necessità di riconoscere quale sia la corretta rappresentazione della realtà in ogni situazione, il ragionare sulle ipotesi implicite e esplicite dietro ad ognuna di esse per circostanziarne il dominio di validità\footnote{Per esempio un'assunzione troppo spesso dimenticata del modello neoclassico è quella di tempo normale, o di assenza di shock esogeni. Trovo più interessante discutere di cosa renda un tempo \textit{normale} e quindi in quali condizioni la rappresentazione neoclassica sia utile, che inscenare una gara su chi abbia ragione \textit{in assoluto}.}.

\section{Sull'Economia}
Abbiamo descritto all'inizio dell'articolo una tensione tra la complessità intesa in senso anti-riduzionista (quella che ho descritto come meta-teoria) e la complessitù intesa come insieme di metodi matematici.
Questa stessa tensione si ritrova nel momento in cui si prova a esplorare il rapporto fra la scienza economica e l'idea di complessità, che è stata recepita sia in senso anti-riduzionista sia come bagaglio tecnico di (nuovi) metodi matematici.
Questa sezione, perciò, non solo fornisce un'analisi dello sviluppo del pensiero economico, ma anche un esempio di come una disciplina possa fare propria la meta-teoria della complessità o, al contrario, limitarsi ad adottare i metodi matematici della complessità.

\marginpar{Una breve storia}Questa analisi identifica in particolare tre relazioni archetipiche tra complessità ed economia, che verranno illustrate e brevemente discusse individualmente dopo una necessaria contestualizzazione storica.

Verso la fine del XIX secolo, nello stesso periodo in cui nascono i metodi matematici della complessità, si sviluppa in economia il paradigma marginalista come approccio di ricerca ispirato ai principi riduzionisti.

Il marginalismo, infatti, cerca di ricondurre lo studio dell'economia alle scelte di ideali agenti economici capaci e volenti di massimizzare dei particolari obiettivi.
In particolare, il problema tipicamente studiato riguarda trovare le condizioni a cui un piccolo numero di agenti (generalmente uno o due) che davanti alla possibilità di effettuare delle transazioni rinuncino perché nessuna di esse è in grado di migliorare (marginalmente) la situazione di entrambi, condizione nota come ottimo paretiano.

Partendo da quest'idea all'apparenza molto semplice la scuola marginalista sviluppa da una parte dei metodi matematici per studiare questa classe di problemi (generalmente ricorrendo all'analisi matematica sul modello della fisica newtoniana), dall'altra un'intera teoria economica deducendo via via proprietà del sistema sempre più complicate.
La teoria marginalista viene costruita, quindi, per aggiunte successive a un nocciolo elementare di situazioni e assunzioni.

La prima archetipica situazione di un'economia di puro scambio --di baratto-- tra due agenti viene via via generalizzata a una teoria della produzione e, molto successivamente, a una teoria dell'aggregato economico.

Non sarebbe però corretto affermare che i (primi) marginalisti non riconoscessero la natura complessa del sistema economico.
\textcite[p. 20]{marshall1988} nei \textit{Principia} scrive che "La società è qualcosa in più della somma delle vite dei singoli"\footnote{"Society is something more than the sum of the lives of its individual members." (traduzione mia).}, riconoscendo l'economia come un sistema complesso.
Ciononostante, seguendo senza dubbio lo spirito del tempo, il tentativo che viene compiuto è quello di spiegare la complessità come proprietà emergente riconducibile analiticamente a poche leggi fondamentali, così come fatto da Boltzmann con la meccanica statistica.

Dopo una fase di relativo declino nella prima metà del '900, l'approccio marginalista diventa indubbiamente egemone nella disciplina economica a partire dagli anni '70.

\marginpar{Sulla definizione di economia}La pervasività dell'operazione egemonica riuscita alla scuola marginalista si può osservare anche --soprattutto-- in come sia mutata la definizione di economia per gli economisti.
Se può sembrare intuitivo che la disciplina economica studi l'economia, o che venga definita con una concettualizzazione simile a "lo studio dei processi di produzione e scambio", questo non è il contenuto della risposta che darebbero la maggior parte degli economisti.

Nel 1932 Robbins pubblica un libro in cui sostiene che "L'oggetto dell'economia è lo studio dei comportamenti umani nella divisione di risorse scarse"\footnote{"The [...] subject of Economic Science [is the study of] the forms assumed by human behaviour in disposing of scarse means." (traduzione mia)} \parencite[p. 15]{robbins2007}.
L'adozione di questa definizione sposta col tempo l'elemento unificante della disciplina dai contenuti ai metodi, permettendo sia una fase di espansione \textit{imperialista} nello studio dei contenuti tipicamente propri di altre discipline \parencite[cfr.][]{stigler1984, lazear2000} sia di escludere dalle posizioni di potere e prestigio quei filoni di ricerca che non utilizzino gli strumenti utili a studiare l'allocazione ottima di risorse finite ovvero che non formulino la propria domanda di ricerca in termini di ottimizzazione vincolata di una certa funzione obiettivo (chiamata utilità, o seguendo Pareto ofelimità).

Nella pratica l'idea marginalista di costruire una teoria economica sulla base di poche assunzioni riguardo il comportamento umano e le scelte che gli agenti affrontano ha sostituito lo studio dell'economia reale come elemento identitario della disciplina economica.

\marginpar{La critica di Lucas}Il riconoscimento della complessità del reale e la sua riconduzione all'interno del paradigma riduzionistico sono presenti anche in uno degli articoli che hanno fondato la macroeconomia moderna.

Nel 1976 Lucas scrisse che "Dato che la struttura di un modello econometrico si basa sulle regole degli agenti per ottenere una decisione ottimale, e che queste regole variano sistematicamente quando occorrono dei cambiamenti nella struttura delle serie rilevanti per colui che deve prendere la decisione, allora ogni cambiamento nelle politiche altererà sistematicamente la struttura del modello econometrico"\footnote{"Given that the structure of an econometric model consists of optimal decision rules of economic agents, and that optimal decision rules vary systematically with changes in the structure of series relevant to the decision maker, it follows that any change in policy will systematically alter the structure of econometric models." (traduzione mia).} \parencite{lucas1976}.
In altre parole Lucas riconosce che ogni modello quantitativo in economia che si basi su dati reali ha un dominio di validità limitato a quel periodo di tempo in cui permangono le norme socio-culturali --le istituzioni-- per cui è stato immaginato.
Nel fare ciò , inoltre, descrive chiaramente la necessità di riconoscere e descrivere due diversi livelli sulla scala dell'aggregazione (le scelte individuali e il comportamento delle variabili aggregate) e le interazioni fra di essi. Cioè, viene resa esplicita la necessità di integrare microeconomia e macroeconomia.

Volendo rimanere però nel paradigma riduzionista la macroeconomia neoclassica che si sviluppa a partire dalla critica di Lucas si pone il problema di ricondurre le leggi degli aggregati economici alle leggi microeconomiche fondamentali (ovvero di microfondare la ricerca macroeconomica), precludendosi la possibilità di immaginare dei modelli che descrivano solo il livello aggregato ma che siano al contempo sufficientemente flessibili da accomodare i cambiamenti nelle istituzioni\footnote{Non sono sconosciuti alla teoria economica i modelli di isteresi che pur presentando due diversi comportamenti macroscopici, possono essere modellizzati da una singola equazione dipendente da un parametro. Generalizzando l'intuizione, la teoria delle biforcazioni è uno strumento che permette di rappresentare cambiamenti macroscopici non-lineari (o non intuitivi) sostituendo alla microfondazione un numero limitato di parametri che astraggando i cambiamenti nelle istituzioni.}.

Il problema può sembrare a prima vista lo stesso affrontato da Boltzmann un secolo prima di ricondurre le leggi macroscopiche della termodinamica alle leggi microscopiche della dinamica molecolare, e un filone di ricerca noto come Econofisica ha provato per anni, senza risultati risolutivi, ad applicare i metodi della meccanica statistica sviluppati da Boltzmann al problema della microfondazione senza successo\footnote{L'elemento mancante in economia per poter tracciare un parallelo utile con la meccanica statistica è la presenza di una quantità che si conservi come l'energia per la fisica. Osservando alcune similitudini questa grandezza dovrebbe avere l'unità di misura di un valore ma nessuna teoria economica riesce a fornire un \textit{principio di conservazione del valore totale} né misurato in termini monetari né misurato in termini di tempo (come valore lavoro).}.
La soluzione perseguita è invece, almeno fino a tempi recenti, quella di identificare i due livelli, cioè descrivere gli aggregati \textit{come se} fossero singoli agenti, descrivere il livello macroscopico \textit{come se} fosse quello microscopico. Questa strategia di approssimazione permette di ottenere dei modelli riduzionisti e analiticamente risolvibili, ma genera una serie di paradossi nel momento in cui si cerca di esplorare il nesso tra l'agente rappresentativo (dell'aggregato) e i singoli agenti economici reali \parencite{kirman1992}.

\marginpar{La prospettiva di Santa Fe}Più o meno contemporaneamente, nel 1987 viene organizzato all'istituto per lo studio dei sistemi complessi di Santa Fe un primo workshop per indagare il possibile ruolo della nascente teoria della complessità nella disciplina economica che vede la partecipazione, tra gli altri, di Anderson, Arrow e Arthur \parencite{fontana2010a}.
\citeauthor{fontana2010a} riporta che "[Arrow] non si aspettava la nascita di un approcio totalmente nuovo: l'impianto teorico generale sarebbe dovuto rimanere invariato, e il ruolo per la `nuova economia', arricchita dalla cooperazioni con fisici e biologi, sarebbe dovuto essere di migliorare lo status quo ante."\footnote{"[Arrow] is not expecting the birth of an entirely new approach: the general framework should remain as it is, with the role for the `new economics', enriched by cooperation with physicists and biologists, being to improve the status quo ante." (traduzione mia)} \parencite{fontana2010a} senza quindi mettere in discussione il metodo deduttivo-riduzionista\footnote{Adottare un approccio deduttivo-riduzionista è fondamentale per molti economisti perché è il metodo delle scienze naturali (e soprattutto della fisica), che permette nella loro visione di elevare l'economia a uno stato di scienza più matura (e in qualche modo migliore in quanto più epistemologicamente solida) delle altre scienze sociali.}.

La direzione del programma per lo studio dell'economia al Santa Fe Institute (SFI) è però assegnata negli anni successivi non ad Arrow ma ad Arthur \parencite{fontana2010a}, che imposta lo sviluppo di un paradigma nuovo che accogliendo l'intuizione di Anderson di porre l'attenzione sulle relazioni tra enti piuttosto che cercare una teoria unificante, si pone come alternativo {--eterodosso--} rispetto al paradigma neoclassico (riduzionistico) \parencite{arthur2021, fontana2010}.
Questo filone di ricerca, noto come economia della complessità o prospettiva di Santa Fe, ha prodotto negli anni numerosi lavori singoli, capaci di indagare temi e impiegare metodi (come le simulazioni computazionali) generalmente trascurati nella disciplina, senza produrre però, per sua stessa natura, una teoria unificante.

\marginpar{Altre recenti determinanti}In tempi recenti altri tre fenomeni hanno interessato l'economia influenzando molto il suo rapporto con la complessità.

Il primo è la diffusione dei calcolatori che ha fatto sì che a fianco alle soluzioni analitiche si cominciassero ad accettare anche soluzioni ottenute con strumenti computazionali \parencite{cherrier2023, backhouse2016}, per esempio aprendo la strada a modelli macroeconomici che abbandonano in parte l'agente rappresentativo per includere un certo grado di eterogeneità (i modelli HANK).

Il secondo è il progressivo spostamento della disciplina dalla ricerca puramente teorico-model\-listica alla ricerca empirica, che basa i risultati su di un'accurata analisi dei dati disponibili \parencite{cherrier2018, backhouse2017}.
Questa svolta neo-empirica è estremizzata da alcuni economisti che rifiutano la necessità di una teoria economica (anche a seguito dei limiti emersi con le crisi dell'inizio del XXI secolo) ritenendo l'analisi dei dati sufficiente a far emergere i rapporti di causa ed effetto sottesi ai fenomeni economici.

Il terzo è la progressiva specializzazione degli economisti che ha portato alla nascita di comunità di ricerca indipendenti tra loro che condividono però un'origine comune nella teoria neoclassica. Questo fenomeno è stato etichettato da alcuni come \textit{mainstream pluralism} \parencite{cedrini2018, davis2006, davis2019a} per sottolineare che esiste una pluralità di programmi di ricerca che convivono tra di loro e convivono con il (e anzi si nutrono del) paradigma dominante --neoclassico-- che li ha preceduti.

Le prossime sezioni saranno destinate a descrivere tre relazioni archetipiche tra complessità ed economia con cui è possibile riflettere sia sullo stato attuale della disciplina economica sia sul ricevimento generale nella scienza dell'idea di complessità.
La prima relazione archetipica affonda le proprie radici nella svolta neo-empirica e fa uso degli strumenti matematici della complessità. La seconda è quella che delineava Arrow e che può essere intesa come una delle cause della frammentazione descritta dal \textit{mainstream pluralism}, che definisco post-neoclassica e fa anch'essa uso degli strumenti matematici della complessità.
La terza partendo dalla prospettiva di Santa Fe riesce a fare propria la complessità come meta-teoria.
Lo scopo di questi paragrafi è, come anticipato, quello di esemplificare e mettere in luce diversi modi in cui la complessità può entrare in relazione con una disciplina, evidenziandone opportunità e rischi.

\subsection{Complessità neo-empirica}
La prima delle tre relazioni archetipiche è quella neo-empirica.

Come detto, il filone di ricerca neo-empirico in economia si concentra sul costruire dei metodi che permettano di trovare relazioni di causalità nei dati, indipendentemente dallo specifico dominio di ricerca. Tra di essi trovano ampio spazio generalizzazioni delle regressioni lineari (come i metodi \textit{difference-in-differences}) e metodi che cercano di manipolare il campione per ottenere coppie di osservazioni da confrontare (come nei \textit{randomized control trials} o nell'uso di procedure di \textit{matching}).

I metodi matematici della complessità offrono miglioramenti per queste due classi di metodi attraverso le tecniche di \textit{machine learning} sviluppate negli anni, che permettono di individuare relazioni non lineari e di trasformare la rappresentazione del campione per migliorare le procedure di accoppiamento delle osservazioni.
Inoltre, la teoria della complessità ha utilizzato negli anni diversi enti matematici per descrivere i dati da analizzare, sviluppando al contempo tecniche specifiche per lo scopo.
Un esempio di queste diverse rappresentazioni sono le reti che hanno permesso di analizzare i dati tenendo conto delle relazioni di mutua interdipendenza delle osservazioni.

\marginpar{La "Complessità economica"}Il filone di ricerca della "complessità economica" \parencite{hidalgo2021} è un esempio interessante di come uno strumento tipico della teoria della complessità (le reti) ha permesso di rappresentare un problema economico (il livello di sviluppo di un'economia) senza ricorrere significativamente alla teoria economica.
L'ipotesi di fondo è che le economie più avanzate producano ed esportino prodotti complessi, e che i prodotti più complessi siano prodotti dalle economie avanzate.
Rappresentando i flussi di import ed export globali per classe di prodotto come una rete, in cui i nodi rappresentano i paesi e gli archi i flussi di merci, si può definire un processo di punto fisso che modellizza l'ipotesi.
Il risultato del processo di punto fisso è un indice che classifica il livello di sviluppo delle nazioni.

È affascinante che un modello molto semplice, che usa come premesse teoriche osservazioni piuttosto semplici, riesca attraverso una rappresentazione efficace dei dati a produrre risultati che ben si allineano all'intuizione.
D'altra parta l'assenza di un modello teorico solido impedisce di problematizzare cosa sia un'economia sviluppata, perché la definizione che se ne ricava è in un certo senso tautologica (un'economia avanzata esporta i prodotti che esportano le economie avanzate), non lasciando spazio all'analisi dei processi storici (ogni prospettiva diacronica, e quindi una scala temporale, è assente se non nel confronto di indici calcolati su periodi differenti) e dei contesti istituzionali che influenzano lo sviluppo economico.

I dati che descrivono i processi economici spesso non sono di eccellente qualità: la numerosità è spesso ridotta, le misure non sono ben definite e il campionamento incompleto.
Tutto ciò rende i dati rumorosi, dando la possibilità ad algoritmi molto flessibili (come in genere sono quelli di \textit{machine learning}) di inferire regolarità che non sono davvero presenti nei fenomeni misurati, ma sono artefatti causati da come i dati sono stati raccolti e analizzati.

L'illusione che eliminare gli schemi teorici dall'analisi dei dati possa produrre una conoscenza oggettiva si scontra con la possibilità di inferire relazioni spurie o non generalizzabili al di fuori del campione analizzato (ritornando al problema della contestualità della conoscenza), problemi che una guida teorica è in grado di ridurre, indirizzando l'analisi verso gli elementi più importanti (distinguendoli dal rumore) e dando senso ai risultati ottenuti.

D'altra parte è impossibile negare che un più ampio ventaglio di metodi matematici sempre più sofisticati ha permesso e permette studi empirici impossibili con i metodi storicamente propri della disciplina economica, permettendo di rappresentare i fenomeni in modi nuovi e più adeguati.

\subsection{Complessità post-neoclassica}
La seconda relazione è quella post-neoclassica.

La teoria neoclassica ha accumulato negli anni (come ogni teoria scientifica di lunga data) una serie di controesempi e di domande di ricerca difficilmente affrontabili con il suo insieme tipico di assunzioni.

Per ovviare a questo problema la scelta di alcuni economisti non è di rifiutare la teoria neoclassica sostituendola con una delle alternative eterodosse o con l'impostazione neo-empirica, quanto piuttosto di rilassare alcune (poche) assunzioni per affrontare una specifica classe di problemi, mantenendo lo stesso approccio quanto più possibile riduzionista-deduttivo\footnote{\textcite{kanazawa2021} paragona questi tentativi all'aggiunta degli epicicli nella tarda teoria geocentrica e credo che il paragone colga bene lo spirito. Il modello diventa più complicato per non dover rimettere in discussione la teoria precedente o accettare un sistema pluralista.}.

I metodi matematici della complessità sono, in alcuni casi, gli strumenti che permettono di rilassare le assunzioni ottenendo lo stesso una soluzione trattabile analiticamente (o computazionalmente).

\marginpar{La teoria dei giochi}Un esempio importantissimo nella storia della microeconomia è la teoria dei giochi iterati. L'assunzione che viene rilassata è che tutti gli agenti posseggano una conoscenza perfetta del contesto, permettendo che invece essa si sviluppi e migliori attraverso la ripetizione del gioco. Allo stesso tempo si introduce nel modello la scala temporale che permette di distinguere l'azione ottima nella singola iterazione con l'azione ottima nell'insieme del gioco.
Così facendo si riescono a ricondurre alla teoria neoclassica situazioni reali in cui, ad esempio, gli agenti mostrano comportamenti collaborativi anziché competitivi.

% Si potrebbero cercare dei riferimenti
Tra gli esempi più recenti troviamo anche l'introduzione di funzioni di utilità non lineari da parte dell'economia comportamentale o l'utilizzo di modelli ad agenti computazionali nella teoria dell'innovazione, che permettono di introdurre asimmetrie informative e altre rigidità e di osservare quanto queste soluzioni si discostino da quelle ottenute con le assunzioni e i metodi tipici.

\marginpar{L'economia ambientale}In altri casi è meno evidente quale sia lo strumento matematico che permette la specializzazione della teoria, ma è molto chiaro quale scala venga aggiunta e quali assunzioni rilassate. Per esempio, l'economia ambientale rilassa l'ipotesi che il prezzo sia indicatore di squilibri di mercato per cercare di assegnare un valore (monetario) anche a beni ed esternalità che non vengono scambiati in un mercato.
Viene cioè introdotta una seconda nozione di prezzo che introduce una scala della scambiabilità dei beni di cui dobbiamo considerare contemporaneamente i due livelli dei beni scambiabili (per i quali valgono le assunzioni tipiche sulla formazione dei prezzi e quindi sulla definizione del valore) e dei beni non scambiabili (per i quali è necessario introdurre una differente nozione di prezzo e di valore).

In entrambi i casi (ovvero il ricorso ai metodi matematici della complessità o l'introduzione di alcuni nuovi livelli su alcune scale) però gli elementi di complessità introdotti sono funzionali al mantenimento (se non al salvataggio) dell'approccio riduzionista.
Questo archetipo di relazione può essere usato anche per interpretare il tentativo (riuscito) di Boltzmann di sviluppare degli strumenti matematici che permettessero di spiegare nuovi fenomeni (la termodinamica) con la vecchia teoria (la dinamica molecolare) senza abbandonarne l'approccio riduzionista.

\subsection{Complessità meta-teorica}
La terza relazione archetipica è quella che, di fatto, imposta la ricerca economica seguendo i principi che ho delineato per intendere la complessità come una meta-teoria\footnote{Evidentemente nessun autore o tradizione può essersi riconosciuto in passato nella meta-teoria della complessità, ciò sarebbe anacronistico considerato che questa espressione sta venendo introdotta per la prima volta in questo articolo. Ci sono però ricercatori che, come già detto, hanno inteso la complessità in senso anti-riduzionista, riconoscendosi per lo più nella prospettiva di Santa Fe o nell'economia della complessità. Consapevole dell'abuso che ciò compie al tempo storico, in questo paragrafo userò il concetto di meta-teoria della complessità per indicare tutto ciò che si allinea a quanto è stato descritto nella prima sezione, in particolare la creazione di modelli ritagliati appositamente sulla singola domanda di ricerca.}.

\marginpar{Economia della complessità}Questa relazione è la più difficile da mettere a fuoco perché, come già detto, non ha prodotto un corpus di conoscenza coeso e facilmente identificabile, quanto più una collezione di lavori indipendenti prodotti da ricercatori che si identificano nella prospettiva di Santa Fe o nell'economia della complessità ma che non costituiscono una comunità di ricerca coesa.

Nello spirito di quanto esposto, questi lavori si concentrano generalmente su di una singola domanda di ricerca sviluppando un modello \textit{ad hoc}, senza dare vita a un flusso di lavori simili tra loro che possano vicendevolmente citarsi, creando un capitale relazionale che permetta di diventare riconoscibile come nicchia di ricerca e di entrare con successo nelle dinamiche di potere alla base dei processi di reclutamento. Un'ottima panoramica è stata pubblicata da \textcite{arthur2021}.

Questo credo sia in parte una caratteristica intrinseca dell'approccio, in parte una strategia di sopravvivenza.
A ogni nuova domanda di ricerca (per quanto simile alle precedenti) corrisponderanno delle scale e delle ipotesi nuove, che necessariamente produrranno dei modelli diversi e richiederanno, inoltre, il tempo di mettere in discussione le assunzioni usate precedentemente, cose che non accadono quando viene adottato un approccio riduzionista.
D'altra parte, non esistendo grandi collaborazioni in questo filone di ricerca, con l'eccezione del Santa Fe Institute dove comunque i ricercatori residenti sono molto pochi, la produzione di articoli si concentra su progetti a breve termine che possano essere portati a compimento da una singola persona o da un piccolo gruppo in tempo per la successiva ricerca di una posizione, di un finanziamento o procedura di valutazione.

% La metateoria della complessità in essenza ci dice che l'economia della complessità non esiste. Giusto? 
Inoltre, l'economia della complessità include un insieme di ricercatori che condividono un'idea molto generale dell'approccio con cui impostare la ricerca, mentre la maggior parte delle altre nicchie di ricerca si identifica con un singolo tema o un singolo metodo.
Assumendo la complessità come meta-teoria, allora, bisogna riconoscere che quella che oggi riconosciamo come economia della complessità non è una teoria economica in senso proprio, ma piuttosto un insieme di metodi, persone e idee blandamente legate fra loro.
In questo senso, nel momento in cui la meta-teoria della complessità dovesse diffondersi nello studio dell'economia sarebbe inevitabile assistere alla fine dell'economia della complessità come programma, mentre i lavori che essa ha prodotto verrebbero inclusi in diversi programmi di ricerca omogenei per temi o metodi che possano effettivamente produrre una teoria del reale.

D'altra parte, soprattutto negli ultimi anni in cui la svolta neo-empirica ha reso meno importante il quadro teorico di riferimento, lavori che seguono la meta-teoria della complessità sono stati accolti in altri filoni di ricerca.
Questo perché nella sua generalità, la meta-teoria della complessità è in grado di costruire e dialogare con la conoscenza precedente, con l'unica accortezza di usarla laddove sia coerente nelle sue assunzioni con le assunzione fatte nella fase di analisi e approssimazione del problema.

\marginpar{Eterodossie}La meta-teoria della complessità è necessariamente eterodossa \parencite{fontana2010} non tanto perché rifiuti nella loro totalità i risultati dell'approccio neoclassico, quanto perché ne rifiuta la ricerca di una teoria universale e indipendente dal contesto.
Per i fenomeni che corrispondono alle assunzioni della teoria neoclassica essa è, dal punto di vista della meta-teoria della complessità, valida, mentre vanno cercate delle alternative per gli altri fenomeni\footnote{L'esempio che ho in mente è la concorrenza oligopolistica tra grandi aziende che è lecito assumere abbiano un comportamento ottimizzante (i profitti), controllino tutte le informazioni disponibili e possano variare in maniera continua il numero di lavoratori e di merce prodotta. Finché queste condizioni sono coerenti con il problema da studiare, credo sia utile fare leva sulla grande quantità di risultati che la teoria neoclassica ha prodotto. D'altra parte è lecito interrogarsi sull'utilità di questi risultati quando le assunzioni non descrivono più il problema di interesse.}.

In alcune parti della galassia eterodossa credo ci sia oggi lo spazio per introdurre l'idea di una meta-teoria della complessità con l'aspettativa che il metodo di lavoro che propone venga fatto proprio da altre tradizioni di ricerca.
In particolare Marc Lavoie nel suo libro di testo sull'economia post-keynesiana \parencite{lavoie2022}, descrive le eterodossie come ispirate da principi di realismo, olismo e organicismo \parencite[][p. 12]{lavoie2022}, nonché dalla necessità di dialogare con altre tradizioni di ricerca\footnote{L'economia post-keynesiana stessa può facilmente essere descritta come una collezione di approcci differenti e complementari tenuti insieme da alcuni generali principi ispiratori che non raggiunge però una monoliticità teorica.}. "Realismo" è la richiesta che le assunzioni usate siano coerenti con la realtà degli oggetti di studio, "olismo" e "organicismo" sono la richiesta di considerare le interazioni tra gli agenti economici e il contesto in cui esse avvengono invece di ricorrere a rappresentazioni archetipiche eccessivamente stilizzate, e la necessità di dialogo è il riconoscimento che ogni teoria si sviluppa per rispondere ad alcune domande mentre si ritrova sprovvista degli strumenti per indagare altre classi di fenomeni.
Queste caratteristiche sono facilmente riconducibili alla descrizione che ho dato di una meta-teoria della complessità, che si concentra sulla descrizione dei fenomeni (reali) nel loro contesto, riconoscendo la necessità di una pluralità di descrizioni per cogliere tutti gli aspetti di un sistema.

\section{Sulla realtà}
\marginpar{Quale realismo?}La descrizione fin qui si è svolta sul piano teorico-metodologico e storico, senza concentrarsi particolarmente sulla pratica scientifica futura.
È lecito quindi chiedersi quanto fare ricerca facendo propri i principi di una meta-teoria della complessità sia realistico.
Questa domanda può essere interpretata in due modi molto diversi: da una parte se può esistere un (eco)sistema della ricerca organizzato facendo propria la meta-teoria della complessità, dall'altra se nell'attuale sistema della ricerca sia possibile per un singolo ricercatore lavorare seguendo i principi della meta-teoria della complessità. E le risposte penso siano opposte.

\marginpar{Il possibile}Un sistema che organicamente fa propria la meta-teoria della complessità è, in essenza, un sistema che pone tanta se non maggiore enfasi sull'impostazione e lo sviluppo della ricerca (la fase iniziale di analisi e di consapevole approssimazione) che non sui risultati prodotti. In cui il lavoro preliminare che il ricercatore fa nel descrivere il sistema da analizzare viene valorizzato.
Allo stesso modo non può che essere un sistema in cui la possibilità di muoversi tra discipline, di utilizzare metodologie non canoniche o di esplorare nuovi problemi non venga penalizzata da metriche estensive e normative di valutazione, in termini di numero di citazioni o di pubblicazioni o di appartenenza a certe comunità di ricerca che si identificano con canoni e stilemi precisi, e, spesso, con liste di giornali di riferimento.

È necessario quindi ridare spazio a lavori che si concentrino sulla fase di analisi e che permettano al ricercatore di condividere con gli strumenti che preferisce (a livello di lingua, linguaggio, media, stilemi da adottare, modalità di diffusione, ...) il processo di analisi e approssimazione, e che questo rientri nella valutazione del suo lavoro, valutazione che può essere fatta solo nel merito e non attraverso metriche bibliometriche.

Questo lavoro è un lavoro lento che richiede che i ricercatori abbiano il tempo di portarlo avanti senza doversi preoccupare della precarietà del proprio contratto. D'altra parte è un lavoro che può essere fatto relativamente poche volte nella propria carriera se, come spesso succede, ci si concentra su una singola domanda di ricerca per molti anni consecutivi.
Chi sarebbe, quindi, più affetto da questo cambiamento sono i giovani ricercatori precari, che già oggi avrebbero bisogno di migliori condizioni materiali di lavoro.

Non trovo preoccupante, invece, la prospettiva che la produzione di ricerca scientifica rallenti se ciò debba essere una conseguenza di un lavoro più preciso e curato, che si allontani dalle logiche performative da catena di montaggio che gli ultimi anni di valutazione della ricerca e allocazione dei fondi hanno incentivato in tutto il mondo.

Il cambiamento più grande che potrebbe venire introdotto è però sul piano epistemico.
Il processo di analisi e approssimazione coinvolge il ricercatore come soggetto agente e si basa sulle sue esperienze e sensibilità.
L'impatto delle differenze di genere o etnia è stato a lungo ignorato perché non appartenevano all'esperienza dei ricercatori, particolarmente in economia.
La contestualità della conoscenza, quindi, deriva anche dalla soggettività della conoscenza, che non è prevista nella visione positivista che popola epistemologicamente il nostro tempo.

Come conseguenza, parte del confronto scientifico dovrebbe spostarsi dalla confutazione dei risultati alla contestualizzazione delle ipotesi, aprendo il problema di come, nell'applicazione tecnica della conoscenza scientifica, far convergere le diverse letture che possono essere date del reale.
In compenso, sarebbe più semplice riconoscere quali determinanti sociali implicite, quali relazioni di potere inespresse, influenzano la creazione di nuova conoscenza e la sua applicazione tecnica.

\marginpar{Il presente}La risposta è quindi che non è un problema adottare la meta-teoria della complessità potendo cambiare le norme sociali e adottando un approccio pragmatico e induttivo al problema dell'applicazione tecnica, mentre sono numerose le norme sociali che ostacolano la sua adozione in un immediato futuro, fra tutte la precarietà del lavoro, l'utilizzo di metriche bibliometriche estensive per la valutazione del lavoro individuale e la progressiva aziendalizzazione dell'accademia, l'insieme di burocratizzazione dei processi e di standardizzazione dei \textit{prodotti della ricerca}\footnote{Fenomeno particolarmente presente in economia data la sua natura fortemente gerarchica e propensa alla standardizzazione. Si pensi all'istituzionalizzazione del \textit{job market paper} come momento conclusivo del dottorato o alla forte direzione che una manciata di dipartimenti e giornali riescono a imprimere a tutta la disciplina \parencite{heckman2020, aistleitner2023, baccini2023a}, creando una differenza di prestigio incolmabile tra i temi, i metodi e gli stilemi adottati dalla comunità dominante e tutti gli altri.}.

Anche la meta-teoria della complessità beneficerebbe di migliori e più inclusive pratiche di ricerca come quelle descritte nella Dichiarazione di San Francisco sulla Valutazione della Ricerca (DORA)\footnote{\url{https://sfdora.org/read/}} o del manifesto di Leida \parencite{hicks2015}, a conferma del fatto che le innovazioni radicali, come quella che qui propongo o quelle che hanno permesso l'evoluzione della scienza, necessitano di un suolo fertile e ricco per germogliare.

\section*{Ringraziamenti}
Tra le figure invisibili dietro questo lavoro c'è la meravigliosa rete informale di amici e colleghi che, nonostante tutto, esiste e resiste a Torino. In particolare Carlo Debernardi e Paolo Babbiotti, che con pazienza continuano a guidarmi nell'esplorazione della letteratura filosofica, Elenora Priori, che per prima ha messo l'accento sull'idea di meta-teoria in una precedente bozza di questo lavoro, e Corrado Monti.
Oltre a loro, questo lavoro è possibile per il supporto e la libertà di ricerca che Eugenio Caverzasi continua a garantirmi.

\begin{refcontext}[sorting=nyt]
	\printbibliography
\end{refcontext}

\end{document}