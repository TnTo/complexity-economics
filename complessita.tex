% !TeX spellcheck = it
\documentclass[a4paper, headings=standardclasses]{scrartcl}

\usepackage[margin=2.5cm]{geometry}
\usepackage{authblk}
\renewcommand{\Affilfont}{\small}
\usepackage[style=authoryear, backend=biber, sorting=nyt, useprefix=true]{biblatex}
\usepackage[autostyle=false, style=english]{csquotes}
\MakeOuterQuote{"}
\usepackage[italian]{babel}
\usepackage[modulo]{lineno}
\linenumbers
\usepackage[hidelinks]{hyperref}

\addbibresource{complexity.bib}

%opening
\title{Complessità come meta-teoria\let\thefootnote\relax\footnotetext{
		Questa versione è un lavoro preparatiorio per un articolo da presentare alla 2023 INEM Conference. \\
		È a sua volta basata su un precedente lavoro presentato alla 24esima ESHET Summer School. \\
		L'ultima versione disponibile di questo lavoro, insieme a una più concisa versione in inglese, è disponibile online \url{https://github.com/TnTo/complexity-economics/}.
}}
\subtitle{Cosa possiamo imparare dall'economia politica}
\author{Michele Ciruzzi\thanks{mciruzzi@uninsubria.it - \url{https://orcid.org/0000-0003-1485-1204}}}

\begin{document}

\maketitle

\begin{abstract}
	Questo articolo è innanzitutto un tentativo d'introspezione, per mettere a fuoco le mie pratiche e i miei desideri nel far ricerca.
	Sebbene non si possa dall'esperienza di un singolo (giovane) ricercatore trarre una legge per un'intera disciplina (un tema, quello del generalizzare, del collegare individuale e collettivo, che ritornerà più volte nell'articolo), molte esperienze sono, in fin dei conti, condivise perché collettive.
	L'analisi storica ed epistemica del rapporto tra complessità ed economia (politica) è un mezzo, se non solamente un pretesto, per parlare di altri modi di fare ricerca oggi, che pongano in essere le condizioni affinché la complessità possa essere usata come paradigma epistemico piuttosto che come "semplice" insieme di strumenti di calcolo.
	Ritengo che un approccio sinceramente complesso all'economia (politica) non possa che essere eterodosso, e che quindi l'aggiunta di elementi "complessi" ai recenti programmi di ricerca \textit{post-neoclassici} sia pretestuosa, che riguardi la complessità esclusivamente su di un piano matematico-computazionale (e per nulla su quello epistemico), come tentativo fuori tempo massimo di salvare le fondamenta neoclassiche della disciplina.
\end{abstract}

\section{La natura sfuggente della complessità}
C'era una volta un mondo ricco di complessità, ma senza una teoria per descriverla. Quel mondo non è poi così in là nel tempo, a ben vedere.

I primi lavori che sviluppano gli strumenti matematici che vengono ancora oggi usati da chi studia i sistemi complessi risalgono alla fine del XX secolo, in particolare gli studi di Poincaré sulla ricorrenza nei sistemi dinamici e quelli di Boltzmann sulla meccanica statistica.

Da lì lo sviluppo dello studio dei sistemi complessi (soprattutto in fisica e, in misura minore, in biologia e informatica) è continuato per quasi un secolo prima che Anderson nel 1972 pubblicasse "More is Different" \parencite{anderson1972}, dando per la prima volta un quadro teorico di riferimento e segnando la nascita della teoria della complessità propriamente detta.

Questa asincronia ha generato, e tuttora genera, un uso degli strumenti matematici senza che spesso siano contestualizzati in un quadro teorico più ampio, guidati più da un empiricalismo radicale che una comprensione del problema come ontologicamente complesso.

Questo perché spesso le stesse pratiche di ricerca si sono sviluppate prima della sistematizzazione della teoria della complessità, che probabilmente non si è ancora conclusa, e si sono consolidati approcci matematici e computazioneli per affrontare singole classi di problemi.

In qualche modo, è ben chiaro quali siano i metodi e i temi da includere nella teoria della complessità, perché possiamo osservare un gruppo di ricercatori che si ritiene una comunità di ricerca perché condividono le stesse origini, ma è meno chiaro cosa li accomuni, perché quei metodi e temi debbano essere inclusi nella teoria della complessità.

Non aiuta, ma è una conseguenza di questo sviluppo diffuso e disorganico, che le definizioni di complessità o di sistema complesso siano numerose e non sempre coerenti tra loro \parencite{holt2011}, spesso perché pensate per cogliere un'aspetto della complessità legato a una specifica nicchia di ricerca o una specifica problematica. Anche perché l'idea di complessità è, negli ultimi ventianni, divenuta essenzialmente pop, un'etichetta senza un significato preciso che si è diffusa ovunque, lasciando tracce in quasi ogni disciplina.

Tutto ciò premesso, nella prossima sezione propongo un'ulteriore definizione di complessità, di cui nessuno sentiva il bisogno, strumentale a questa argomentazione e che cerca di evidenziare il tratto davvero universale della questione, rinunciando a concetti (come l'emergenza, la non-linearità, il confronto fra la parte e il tutto) che assumono significato solo in alcuni ambiti di ricerca, per metodi o oggetti.

\section{Di lenti, nomi e modi}
\subsection{Di lenti, o una metafora}
Le lenti hanno giocato nella storia un ruolo importante nel permettere la produzione di nuova conoscenza. Gli occhiali da lettura, i cannocchiali per l'esplorazione della terra e del cielo, il microscopio che rende visibile ciò che è troppo piccolo per essere visto.

Una singola lente però non è in grado di mostrare tutto: se devi guardare un pianeta non puoi usare un microscopio, se vuoi studiare un batterio non puoi usare un microscopio, se sei miope non puoi usare delle lenti da presbite.

Ma ognuno di essi permette di mettere a fuoco qualcosa di diverso, e con un ipotetico set completo di lenti si può mettere a fuoco tutto. Il punto è capire quale sia necessaria, cosa davvero ci interessa studiare con cura.

Si può, come esperimento mentale, pensare di riuscire a creare una descrizione del sapere così comprensiva, al contempo analitica e sintetica, da rendere ogni altra obsoleta, ma non si discosterebbe molto da una mappa 1:1, sulla cui inutilità hanno scritto, meglio di me, Eco e Borges.

Dobbiamo allora riconoscere che ogni descrizione del reale non può che essere parziale, centrata su qualche elemento di interesse mentre ne tralascia, inevitabilmente, altri.

La complessità è questo: riconoscere che il mondo è composto da troppe cose, legate da troppi nessi e relazioni, per poterle isolare una alla volta o studiarle tutte insieme con precisioni. Invece, si può ragionare su quali siano, di volta in volta, le cose importanti da mettere a fuoco, l'ingrandimento necessario per vedere ciò che mi interessa. Sapendo però cosa non sto vedendo, ed essendo consapevoli del perché si è deciso di lasciare alcune cose fuori dal campo visivo.

L'esperienza di usare un microscopio e guardarci dentro è illuminante: lo stesso vetrino, lo stesso campione, appare completamente diverso a ingrandimenti e su piani focali diversi. Ma è lo stesso oggetto. E non è possibile scegliere un'immagine, una rappresentazione, intrinsecamente vere. Sono tutte vere, possono però essere più o meno adatte a uno scopo, a mettere in risalto una specifica struttura o degli specifici elementi. Ma anche scelta una le altre non scompaiono, rimangono lì, in attesa delle loro domande, delle domande per cui essi sono adatti.

\subsection{Di nomi, o una definizione}
\textit{Un sistema è complesso se deve essere descritto diversamente a diversi livelli di una o più scale.}

Sistema. Complesso. Descritto. Diverso. Livelli. Scala.

Uno per volta.

Complesso non ha bisogno di molte spiegazioni: è un paper sulla teoria della complessità e in qualche modo deve rientrare nella definizione.

Sistema è più interessante, intanto perché l'idea che "la teoria della complessità studi i sistemi complessi" è talmente diffusa da poter essere considerata quasi tautologica, e poi perché comincia a descrivere ciò che vorremmo studiare.
Un sistema è necessariamente composto da più enti in relazione (un concetto che piace tantissimo a chi come strumento di elezione ha scelto le reti), è antitetico alla monade isolata.

In altre parole, parlando di sistemi implichiamo l'esistenza di moltitudini di enti, forme, descrizioni, configurazioni, \dots

Descrivere è forse l'atto primigenio della pratica scientifica e un passo necessario nella maturazione di una teoria. Descrivere ci riporta anche a contatto con la realtà, introduce la necessità di avere un oggetto di studio riconoscibile, delinabile, in altre parole riporta il realismo nello sviluppo della teoria.

Per molte discipline che la conoscenza debba essere realistica, cioè che non deve contraddire l'esperienza del reale, è un assunto fondante e condiviso, ma non in tutte è così, soprattutto nelle scienze sociali dove diviene innegabile (per quanto continuamente negato) che le forme della rappresentanzione del reale sono politiche.

Descrivere inoltre implica un'analisi accurata dell'oggetto di studio, senza, almeno in primo luogo, scorciatoie e apprrossimazioni. In alcuni contesti è possibile che si possa descrivere con cura e minuzia usando il linguaggio formale della matematica o una lingua che non si padroneggia perfettamente, ma in generale se descrivere (e quindi, forse, comprendere) è un aspetto fondamentale per esplorare ciò che è complesso, un ritorno all'uso del proprio linguaggio naturale (e forse anche l'uso di una multi-medialità) diventa una pratica imprescindibile.

Scala, come quella su una mappa geografica o di una linea del tempo.
Scala è il vero concetto centrale della definizione, quello che ci permette di generalizzare numerose altre intuizioni come l'emergenza, la non linearità, il caos deterministico, l'ergodicità.

Ma che cos'è una scala? Per lo spazio e il tempo l'intuizione è abbastanza chiara. Voglio studiare qualcosa che avviene a livello atomico, che rigurda le dimensioni che normalmente esperiamo, o il moto delle galassie? La storia di una religione si sviluppa su secoli, quella di una farfalla in giorni, quella di un positrone in frazioni di secondo.

Allo stesso modo però se studiamo gli esseri umani e le loro relazioni possiamo concentrarci su di un individuo, una piccolissima comunità (una famiglia, dei coinquilini, una squadra di pallacanestro), o addirittura un quartiere o un intero stato.

Una scala è, alla fine, un aspetto del sistema, una sua area semantica e concettuale, che possa essere analizzato da punti di vista diversi e con differenti livelli di dettaglio.

Livello, quindi, sono i vari modi di esplorare una scala, è il pezzetto della scala su cui ci si concentra ignorando il resto, perché è impossibile tenere in conto ogni dettaglio, ogni sfacettatura e al contempo avere una descrizione chiara e maneggiabile di ciò che si vuole studiare.

Ma concentrarsi su un livello, ovviamente, non implica che gli altri non esistano, solo che il fenomeno cui si è interessati ha effetti trascurabili ai livelli ignorati, rispetto agli effetti che si monstrato ai livelli tenuti in considerazione.

In questo emerge tutta l'importanza della descrizione: devo descrivere e comprendere il sistema al punto di capire quali effetti agiscono su quali livelli di quali scale, per poter approssimare e astrarre il problema fino al punto che diventi trattabile.

\subsection{Di modi, o una metodologia}
La definizione cha ha aperto il paragrafo precendente è, per essere la definizione di una teoria, strana: è operativa, dice poco e in modo indiretto degli assiomi fondanti della teoria, non la pone in un contesto storico o culturale più ampio.
La ragione penso sia, come discuterò nel prossimo paragrafo, che la teoria della complessità non sia una teoria, ma, piuttosto, qualcosa a metà strada tra una metodologia e un approccio epistemico.

Le prescrizioni normative della definizione data sono per lo più di carattere metodologico.

È necessario individuare il sistema di interesse e il fenomeno che si vuole indagare, darne una descrizione analitica sufficientemente dettagliata da riconoscere le scale lungo cui il sistema si sviluppa, individuare su ogni scala a quali livelli si osserva il fenomeno d'interesse.

È un processo a togliere, in cui da una comprensione del generale si semplifica, necessariamente consci di quello che si sta rimuovendo, fino a considerare il minimo numero di dettagli (di livelli) necessari a comprendere il fenomeno.

È evidente, quindi, che questa "metodolgia della complessità" suggerisca un metodo di lavoro opposto a e inconciliabile con il diffuso riduzionismo, che invece lavora per addizione, nel tentativo di spiegare quanti più fenomeni come conseguenze di pochi principi primi, aggiungendo le necessarie variazioni a poche fondamendali regole.

In qualche modo invece, ciò che sto provando a delineare è una metodologia del particolare e dell'unico, del riconoscere che i tratti importanti di un sistema sono molteplici e differenti a seconda dello scopo, della domanda. Ed è nel riconoscere il ruolo organico del particolare all'interno del generale, invece che rappresentare invece il particolare come variazione sul tema del generale, che il paradigma della complessità può divenire uno strumento di dialogo tra discipline, una categoria di pensiero applicabile alla realtà generale e, forse, il motore di un cambio di paradigma scientifico.

Questa sensibilità al particolare invita anche a recuperare i metodi qualitativi della ricerca affianco a quelli quantititativi, perché in grado di esplorare lo stesso fenomeno a livelli diversi, dandone una conoscenza non contraddittoria, come può sembrare, ma sfacettata e complementare.

È una prospettiva di cura e democratizzante, perché restituisce all'esperienza umana individuale una dimensione propria non come variazione di un tipo generale, ma come portatrice di particolarità proprie osservabili solo su ci si pone al giusto livello della giusta scala.

\subsection{Meta-teorie}
Se la teoria della complessità non è una teoria, che cos'è?

Il termine migliore che ho trovato è meta-teoria: una teoria su come sviluppare teorie. Perché è più che una metodologia, avendo anche un contenuto epistemico importante, ma non è una teoria, in quanto incapace \textit{in sé} di produrre predizioni o ragionamenti.

Per provare a chiarire cosa intendo con meta-teoria, proverò a fare un esempio riferendomi a quella che considero un'altra meta-teoria sicuramente più diffusa e riconoscibile: gli studi di genere.

L'intuizione teorica alla base degli studi di genere è, semplificando all'osso, che qualunque sia il campo di indagine bisogna assumere che, fino prova contraria, il genere delle persone cui lo studio si riferisce deve essere tenuto in considerazione a meno che non ci siano validi motivi per ignorarlo.

Questa intuizione non ci permette di per sè di sviluppare teorie, non è sufficiente, ma offre delle prescrizioni normative su come impostare queste teorie, su quali elementi dovrebbe e non dovrebbero essere presenti nei metodi e modi con cui le teorie vengono sviluppate e applicate.

A ben guardare, gli studi di genere (come anche le teorie che si rifanno all'intersezionalità) rientrano pienamente nel paradigma della complessità che sto descrivendo: ci chiedono di pensare se sulla scala delle caratteristiche fisiche (o culturali) degli esseri umani il livello del sesso (o del genere) sia rilevante, cioè se possiamo (o vogliamo consapevolmente, approssimando,) escludere che il fenomeno che stiamo studiando si manifesti diversamente tra uomini e donne.

È possibile che questa sessione si possa riassumere con il concetto di epistemologia, ma non credo di avere le capacità di sostenere un'argomentazione precisa per un'"empistemologia della complessità". Per questo motivo seguo una definizione maggiormente normativa e operativa, descrivendo una teoria sulle teorie.

\section{Due accezioni di complessità in economia}
\subsection{Una storia anacronistica}

\subsection{La prospettiva di Santa Fe}

\subsection{La prospettiva post-neoclassica}

\subsection{Una parola sul pluralismo}

\section{Praticare la complessità ai tempi del neoliberismo}
\subsection{La catena di montaggio nel lavoro accademico}

\subsection{La legittimazione sociale del lavoro accademico}

\subsection{La necessità per la ricerca di lentezza e artigianalità}
I want to emphasize that a sincerely complex approach puts at least as much emphasis on the processes than on the results, if not more. How a research question is chosen and how a model is tailored (and so the reasoning behind it) is a fundamental part of how research is done and should be communicated and valorized on its own, where results remain a useful appendix of doing science.

% Multi-medialità

\printbibliography

\end{document}
