% !TeX spellcheck = en_GB
\documentclass[a4paper, headings=standardclasses]{scrartcl}

\usepackage{authblk}
\renewcommand{\Affilfont}{\small}
\usepackage[style=authoryear, backend=biber, sorting=nyt, useprefix=true]{biblatex}
\usepackage[autostyle=false, style=english]{csquotes}
\MakeOuterQuote{"}
\usepackage[english]{babel}

\addbibresource{complexity.bib}

%opening
\title{On Complexity as a Meta-Theory}
\subtitle{A Perspective from Economics}
\author{Michele Ciruzzi\thanks{mciruzzi@uninsubria.it - https://orcid.org/0000-0003-1485-1204}}

\begin{document}
	
	\maketitle
	
	\begin{abstract}
This paper was born by the need of introspection, of understanding how I (wish to) do research. Even if we cannot use the experience of a single (early-stage) researcher to describe an entire discipline (a theme, the relation between a part and the whole, which recurs in the paper), many issues are reflected in individual experience.
By analysing the historical and epistemic relation between Economics and Complexity, I want to explore an alternative way to do research, by looking at the condition required to embrace Complexity as epistemology rather than a mathematical toolbox. 
I argue that a sincerely complex approach to Economics must be heterodox and, similarly, the recent inclusion of "complex" elements in \textit{post-neoclassical} research programs is only about the kind of mathematics used (and not at all about a change in epistemology) and an out-of-time attempt to save the neoclassical core of the discipline.
	\end{abstract}
	
	\section{The Elusive Nature of Complexity}
Once upon a time there was a world full of complexity but without a theory to describe it. This world is indeed not so far in the past.


	
	\section{A Metaphor, a Definition, a Methodology}
	\subsection{A Metaphor}
	
	\subsection{A Definition}
	
	\subsection{A Methodology}
	
	\section{The two Complexities in Economics}
	\subsection{An Anachronistic History}
	
	\subsection{The Santa Fe Perspective}
	
	\subsection{A Post-Neoclassical Perspective}
	
	\section{Doing Complexity in Neoliberal Academia}
	\subsection{The Serialization of Creative Work}
	
    \subsection{The Social Legitimacy of Academic Workers}

	\subsection{The Need for Slow and Artisan Research}
	I want to empathize that a sincerely complex approach puts at leat as much emphasis on the processes than on the results, if not more. How a research question is chosen and how a model is tailored (and so the reasoning behind it) are a fundamental part of how research is doing and should be communicated and valorized on its own, where results remain a useful appendix of doing science.

	\printbibliography
	
\end{document}
