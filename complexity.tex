% !TeX spellcheck = en_GB
\documentclass[a4paper, headings=standardclasses]{scrartcl}

\usepackage{authblk}
\renewcommand{\Affilfont}{\small}
\usepackage[style=authoryear, backend=biber, sorting=nyt, useprefix=true]{biblatex}
\usepackage[autostyle=false, style=english]{csquotes}
\MakeOuterQuote{"}
\usepackage[english]{babel}

\addbibresource{complexity.bib}

%opening
\title{On Complexity as a Meta-Theory}
\subtitle{A Perspective from Economics}
\author{Margherita Ciruzzi\thanks{mciruzzi@uninsubria.it - https://orcid.org/0000-0003-1485-1204}}

\begin{document}

\maketitle

\begin{abstract}

\end{abstract}

\section{The Elusive Nature of Complexity}
The study of complexity is always

\section{A Metaphor, a Definition, a Methodology}
\subsection{A Metaphor}

\subsection{A Definition}

\subsection{A Methodology}

\subsection{A Meta-Theory}

\section{The two Complexities in Economics}
\subsection{An Anachronistic History}

\subsection{The Santa Fe Perspective}

\subsection{A Post-Neoclassical Perspective}

\subsection{A Note on Pluralism}

\section{Doing Complexity in Neoliberal Academia}
\subsection{The Serialization of Creative Work}

\subsection{The Social Legitimacy of Academic Workers}

\subsection{The Need for Slow and Artisan Research}
I want to empathize that a sincerely complex approach puts at leat as much emphasis on the processes than on the results, if not more. How a research question is chosen and how a model is tailored (and so the reasoning behind it) are a fundamental part of how research is doing and should be communicated and valorized on its own, where results remain a useful appendix of doing science.

\printbibliography

\end{document}
